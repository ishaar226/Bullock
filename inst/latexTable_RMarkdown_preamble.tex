% latexTable_RMarkdown_preamble.tex
%
% This file may be used when using latexTable() tables in an R Markdown 
% document. It loads all of the packages that may be required to produce 
% latexTable() output. To use this file, the header of your R Markdown
% document should look like this:
%
% ---
% title: "My Document"
% output: 
%   pdf_document:
%     includes:
%       in_header: "latexTable_RMarkdown_preamble.tex"
%     keep_md: true
%     keep_tex: true
% ---


\usepackage{afterpage}    

\usepackage{array} 
  \setlength{\extrarowheight}{7pt}   % adding 1pt of space to every row in every table

\usepackage{booktabs}
  \setlength{\heavyrulewidth}{.08em} % for \toprule and \bottomrule.  .08em is the default.
  \setlength{\belowrulesep}{2.0ex}   % for \cmidrule.  .08ex is the default.

\newlength\captionIndent
  \setlength\captionIndent{.45in}
\usepackage[font=rm, labelfont={it}, justification=RaggedRight, margin={0in}, parskip=0pt]{caption}
  \setlength{\textfloatsep}{30pt plus 2pt minus 2pt}
  \DeclareCaptionLabelFormat{noindent}{\fontsize{12bp}{14.4bp}\selectfont\bothIfFirst{\noindent\hspace*{-\RaggedRightParindent}#1\ }{#2}}
  \captionsetup{labelformat=noindent, aboveskip=15bp, justification=RaggedRight}

\usepackage{float} 

\usepackage[autolanguage, noaddmissingzero, warning]{numprint}
  \npfourdigitnosep % print "6002.54" instead of "6,002.45"

\usepackage{pdflscape}    

\usepackage[document]{ragged2e}
  % Load this package after fonts and font sizes have been established.  (Companion 2e, 106.)
  \setlength\RaggedRightRightskip{0pt plus .6in} %  for hyphenation % .6
  \setlength\RaggedRightParindent{.5in}
  \setlength\JustifyingParindent{0in}

